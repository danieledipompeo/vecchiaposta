
%Sometimes it is a good idea to put domain objects in \texttt{}
%The template and the descriptions are based on the book Applying UML and Patterns: 
%An Introduction to Object-Oriented Analysis and Design and Iterative Development
%(3rd Edition) by Craig Larman.
\begin{usecase}

\addtitle{Caso d'uso:}{UC5 - Effettuare una richiesta di prenotazione} 

%Scope: the system under design
\addfield{Scopo:}{Utente}

%Level: "user-goal" or "subfunction"
\addfield{Livello:}{Utente}

%Primary Actor: Calls on the system to deliver its services.
\addfield{Attore primario:}{Cliente}

%Stakeholders and Interests: Who cares about this use case and what do they want?

%Preconditions: What must be true on start and worth telling the reader?
% \addfield{Preconditions:}{}
%when multiple
\additemizedfield{Precondizioni:}{
  \item Il sito web è navigabile
} 

%Postconditions: What must be true on successful completion and worth telling the reader
\addfield{Postcondizioni:}{La richiesta è avvenuta con successo}
%when multiple
%\additemizedfield{Preconditions:}{}

%Main Success Scenario: A typical, unconditional happy path scenario of success.
\addscenario{Scenario principale di successo:}{
	\item il cliente accede alla pagina web
	\item Il sistema mostra all'utente la pagina home page
	\item l'utente richiede la pagine per le prenotazioni
	\item il sistema mostra correttamente la sezione richiesta
	\item il sistema carica il \textit{form} per l'invio dei dati
	\item l'utente compila i dati del \textit{form} mostrato
	\item l'utente invia i dati
	\item il sistema salva i dati
	\item il sistema invia email all'utente per la corretta ricezione della richiesta
	\item il sistema notifica all'amministratore la richiesta avvenuta
	\item il sistema mostra all'utente la home page
	\item l'utente abbandona il sito
}

%Extensions: Alternate scenarios of success or failure.
\addscenario{Estensioni:}{
	\item[4.a] Pagina richiesta non disponibile:
		\begin{enumerate}
		\item[1.] il sistema non riesce a mostrare la sezione richiesta
		\item[2.] il sistema invia una email al progettista con il report del bug
		\item[3.] il sistema torna al punto 2
		\end{enumerate}
	\item[7.a] Invio errato:
		\begin{enumerate}
		\item[1.] il sistema non esegue l'invio
		\item[2.] il sistema mostra l'eventuale errore all'utente
		\item[3.] il sistema invia una mail al progettista con il report del bug
		\item[4.] il sistema mostra all'utente la pagina di provenienza
		\end{enumerate}
	\item[7.b] Dati inseriti errati:
		\begin{enumerate}
		\item[1.] il sistema non esegue l'invio
		\item[2.] il sistema mostra l'eventuale o gli eventuali dati inseriti errati
		\item[4.] l'utente correggie gli errori
		\item[5.] l'utente esegue il punto 7
		\end{enumerate}
}

%Frequency of Occurrence: Influences investigation, testing and timing of implementation.
\addfield{Frequenza:}{media/alta nelle fasi successive}

%Miscellaneous: Such as open issues/questions
%\addfield{Open Issues:}{}

\end{usecase}

