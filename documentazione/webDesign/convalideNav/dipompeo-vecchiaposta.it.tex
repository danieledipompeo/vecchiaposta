\documentclass[a4paper,12pt,hidelinks]{report}

%%Pacchetti utili anche se non necessari

\usepackage{amsfonts}
\usepackage{amsmath}
\usepackage{latexsym}
\usepackage{tabularx}
\usepackage[italian]{babel}
\usepackage[bookmarks=true]{hyperref}
\usepackage{url}
% \usepackage{subfigure}
\usepackage{epstopdf}
\usepackage[utf8]{inputenc}
% \usepackage[utf8x]{inputenc}
\usepackage{listings}
\usepackage{graphicx}
%-------------------------------------------

% \title{Progettazione sito web\\ ''B\&B La Vecchia Posta''}
% \author{Daniele Di Pompeo \\mat. 226766}
% \annoaccademico{2013-2014}
\begin{document}
  \begin{titlepage}
    \begin{center}
    % Upper part of the page
      \includegraphics[width=0.5\textwidth,keepaspectratio=true]{../../img/logo}\\[1cm]    
      \textsc{\LARGE Covalide prototipo di navigazione}\\[0.6cm]
      \textsc{\LARGE  progetto del sito:\\[0.5cm] ``B\&B La Vecchia Posta''}\\ [2.0cm]

    % Author and supervisor
      \begin{minipage}{0.8\textwidth}
	\begin{flushleft} \large
	  \emph{Autore:} Daniele Di Pompeo \\[0.5cm]
	  \emph{Versione documento: 1.2}\\[0.5cm]
	  \emph{Data emissione del documento: \today}\\[0.5cm]
	\end{flushleft}
      \end{minipage}
    \end{center}
  \end{titlepage}

% \tableofcontents
 
\begin{abstract}
 In questo documento vengono elencate le peculiarità di navigazione del sito web in oggetto che sono state descritte nel documento dei requisiti in allegato
 alla documentazione rilasciata al committente.
 Il prototipo di navigazione è raggiungibile all'indirizzo \url{http://http://www.vecchiaposta.it/nuovosito/web/progettazioneV2/index.html}.
 \par Il documento si riverisce al prototipo di navigazione, quindi si mostrano le considerazioni finali sui requisiti tecnici individuati nel relativo documento.
 Il tema presentato nel prototipo fornisce all'utente tutti gli strumenti per testarne l'usabilità, fornendo una reale e quasi totale navigazione del sito.
\end{abstract}

\section*{Layout generale}
Il protipo del layout utilizzato per la realizzazione del sito web è stato sottoposto al test di usabilità di vari tipi di utenti.
Ai vari utente sono state poste le seguenti domande:
\begin{itemize}
 \item Potresti raggiungere la sezione delle camere?
 \item Sei interessato alle attività proposte, sai raggiungere la sezione del sito?
 \item Riesci a scoprire gli eventi che ci sono nella zona?
 \item Cosa per te deve essere aggiunto o eliminato?
\end{itemize}
All'ultima domanda il test non è riuscito a fornire una risposta dettagliata essendo il prototipo presentato è completamente spoglio del comparto grafico.

\begin{itemize}
 \item \textbf{Utente esperto}: l'utente esperto ci ha fornito un feedback positivo sia per quanto riguarda lo stile del sito sia per la facile navigazione tra le sezione
 dello stesso. In particolare è riuscito subito a raggiungere la sezione desiderata ed ha apprezzato la struttura delle pagine.
 \item \textbf{Navigatore esperto}: facendo utilizzare il prototipo del sito web ad un navigatore esperto (utente che trascorre alcune ore giornaliere su internet), il sito 
 risulta essere abbastanza navigabile non potendo suggerire migliorie sul comparto grafico il test verrà riproposto a sviluppo ultimato.
 \item \textbf{Navigatore inesperto}: forse il test più importante per assicurare la buona riuscita della progettazione del sito. I risultati ottenuti si possono considerare 
 accetabili in relazione al nostro utente. 
 \par Il sito è risultato navigabile nella sua quasi totalità. L'utente non ha avuto grandi difficoltà nel effettuare le operazioni richieste.
\end{itemize}

\section*{Struttura dei menu}
La struttura del menu risulta essere completamente visibile sia alla risoluzione scelta per la progettazione, sia ad una risoluzione più bassa.
Sono stati evitati scroll orizzontali nelle risoluzioni 1024px (\autoref{fig:nav1024}) e 1200px (\autoref{fig:nav1200}) (risoluzione di base) che coprono quasi la totalità 
delle configurazioni degli utenti che navigano su internet non attraverso dispositivi mobile.
\\Per quanto riguarda risoluzioni più basse e dispositivi mobile, si rimanda ad una versione successiva dello sviluppo.
\par La struttura del menu mantiene una proporzionalità accettabile con i contenuti informativi in entrambe le risoluzioni considerate. 
\\Si mostrano due screenshot alle due risoluzioni.
\begin{figure}[h!]%
    \includegraphics[width=0.9\textwidth,keepaspectratio=true]{../../img/nav1024}
    \centering
    \caption{Prototipo: risoluzione 1024px}%
    \label{fig:nav1024}%
\end{figure}

\begin{figure}[h!]%
    \includegraphics[width=0.9\textwidth,keepaspectratio=true]{../../img/nav1200}
    \centering
    \caption{Prototipo: risoluzione 1200px}%
    \label{fig:nav1200}%
\end{figure}
Per quanto riguarda la struttura del menu si è scelto di progettarla sotto il paradigma del menu a comparsa, che ci ha permesso di mantenere l'ingombro della barra di navigazione
al minimo e quindi di non assegnare ulteriore spazio ad un menu laterale per accedere alle sottosezioni.

\section*{Navigazione}
Per la progettazione delle gabbie logiche (\autoref{fig:navGL}) ci si è concentrati nello sviluppo di una struttura delle pagine che consenta di essere ripetibile nelle varie sezioni.
\par Le sotto sezioni sono state strutturate in modo tale da avere sopra la piega tutte le informazioni utili alla navigazione. 
\\In alto a sinistra si è inserito un'area destinata a contenere uno slider per le immagini relative alla pagina o a contenere la mappa stradale.
Si è utilizzato la restante area per fornire all'utente una descrizione testuale della sezione desiderata. 
\\Per quanto riguarda la struttura della pagina descrittiva delle camere si è scelto di inserire nell'area destra una ``pre-form'' per la richiesta di disponibilità 
e la lista degli optional disponibili nella camera.
\\Le uniche pagine che hanno una struttura differente sono le pagine delle camere. Per queste pagine web si è scelto di inserire al di sopra della piega oltre ad uno slider anche una
``pre-form'' per la prenotazione e l'elenco puntato dei servizi disponibili nella camera.
\\Si riporta uno screenshot della progettazione degli ingombri per una pagina interna del sito.
\begin{figure}[h!]%
    \includegraphics[width=0.9\textwidth,keepaspectratio=true]{../../img/navGabbieLogiche}
    \centering
    \caption{Gabbia Logica: ingombro pagina interna}%
    \label{fig:navGL}%
\end{figure}
\newline 
Inizialmete si era pensato di inserire la ``pre-form'' di prenotazione nella parte sopra la piega di tutte le pagine interne, ma i test hanno riportato esito negativo su tale scelta.
\\Si è progettato allora di inserire nel menu la voce ``Prenota'' al posto della voce ``Dove Siamo'', anche se dai dati statistici di google Analytics risultano esserci pochi click sulla voce del
menu ``Prenota''. Per ovviare a questo problema e risolvere il problema emerso nei test si è scelto anche di inserire un ``widget'' grafico a fianco del testo descrittivo della pagina interna del 
sito web come scorciatoia per arrivare alla pagina ``Prenota''.

\section*{Etichette}
Nel documento dei requisiti, in allegato alla documentazione, alla sezione \textit{2.1.1 - Architettura informativa}, sono state individuate alcune aree semantiche 
che dovevano essere inserite nel sito. 
\par A valle dei test eseguiti sugli utenti la scelta di utilizzare la stessa nomenclatura individuata nello storyboard risulta essere adeguata e non ha sottolineato problematiche tali 
da rendere necessario una riprogettazione delle stesse. 
I vari utenti a cui è stato sottoposto il sito web hanno mostrato una facilità di navigazione del sito e non hanno riscontrato alcuna difficoltà legata all'ambiguità delle
etichette rispetto alla pagina web a cui sono legate.

\section*{Compatibilità browser}
Il sito web risulta essere compatibile con i principali browser sul mercato in particolare è stato testato su:
\begin{itemize}
 \item Chrome
 \item Firefox
 \item Opera
 \item Internet Explorer 10
\end{itemize}
La compatibilità con i precedenti browser ci permette di coprire quasi la totalità dei navigatori ``desktop'' di internet.
\par Per quanto riguarda la compatibilità con browser per mobile, come il browser \textit{safari}, che copre una buona percentuale per quanto riguarda i dispositivi mobile iOS, sarà analizzata successivamente
quando la parte mobile del sito sarà in fase di sviluppo.

\end{document}          